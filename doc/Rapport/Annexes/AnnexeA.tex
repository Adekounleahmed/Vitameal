\chapter{Composition de repas}

\section{Le petit-déjeuner}

\subsection{Composition}

Le petit-déjeuner comporte au minimum les éléments suivants :
\begin{itemize}
	\item une boisson : eau, jus de fruit (100\% fruit, sans sucre ajoutée), lait demi écrémé, café, café décaféinée, thé, tisane, chicorée, ... ;
	\item un aliment céréalier (pain, biscottes, ou autre produit céréalier, ...) ;
	\item un produit laitier (lait, yaourt, fromage ou autre produit laitier, ...) ;
	\item un fruit(fruit cru, jus de fruit, compote, purée de fruit).
\end{itemize}

Le lait est considéré comme une boisson et un produit laitier et le jus de fruit est considéré comme une boisson et comme un fruit.

Selon type de population, le petit-déjeuner peut éventuellement être complété par :
\begin{itemize}
	\item un élément lipidique (beurre, margarine, ...) ;
	\item un élément sucré (confiture, gelée, miel, ...) ;
	\item un élément protidique (jambon, oeuf, ...).
\end{itemize}

\subsection{Restrictions}

Il convient d'éviter les pâtes à tartiner et les pâtisseries contenant plus de 15 \% de matières grasses, c'est à dire :
\begin{itemize}
	\item les viennoiseries (croissant, pain au chocolat, ...) ;
	\item les barres chocolatées ;
	\item les biscuits chocolatés ou fourrés ;
	\item les céréales fourrées ;
	\item les beignets,
	\item les gaufres,
	\item les crêpes fourrées au chocolat,
	\item les gâteaux à la crème ou au chocolat,
	\item les brownies au chocolat et aux noix,
	\item les quatre-quarts,
	\item les gâteaux moelleux chocolatés type napolitain mini-roulé,
	\item les biscuits chocolatés,
	\item les biscuits sablés nappés de chocolat,
	\item les biscuits secs chocolatés,
	\item les galettes ou les sablés,
	\item les goûters chocolatés fourrés,
	\item les gaufrettes fourrées,
	\item les madeleines,
	\item les biscuits secs feuilletés type palmier
	\item les cookies au chocolat.
\end{itemize}

Ainsi que les desserts suivant :
\begin{itemize}
\item les tiramisus ;
\item les crèmes brûlées ;
\item les glaces ou les nougats glacés.
\end{itemize}

La fréquence recommandée est de 3 repas sur 20 repas successifs au maximum.

\subsection{Références}

\ref{docNutrition}, § 3.2.1 (page 17), § 4.2.1.1.4 (page 39)
