%-*- coding: utf-8 -*-
\textcolor[RGB]{46, 116, 181}{\chapter{Initialisation}}
\section{Définition du problème}
L'élaboration de menus dans un hôpital pour la restauration des patients
est une tâche complexe, et doit tenir compte des différentes pathologies
rencontrées. Faute de moyens (temps et argent) seules quelques grandes
lignes de restauration sont retenues; alors qu'idéalement, chaque
patient devrait pourvoir avoir un repas adapté à sa pathologie.

\section{Vision du projet}
\subsection{Solution envisagée}
Mise en place d'un outil informatique permettant d'élaborer les menus
des patients en fonction des profils diététiques, paramétré par le corps
médical.

\subsection{Périmètre}
C'est un diététicien qui renseigne le profil diététique des patients,
sous les directives des médecins. C'est aussi un diététicien qui élabore
les menus des patients. L'outil permettra donc au diététicien d'élaborer
les menus par filtrage des produits correspondants aux profils
diététiques des patients.

\section{Analyse des exigences}
\subsection{Partie prenantes}
\begin{itemize}
\item Participantes~: les diététiciens, le service restauration
\item Concernés~: les médecins, la direction (budget)
\item Impactées~: les patients
\end{itemize}

\subsection{Les besoins}
\begin{itemize}
\item Les diététiciens renseignent les profils diététiques de chaque patient.
\item Les diététiciens élaborent les menus.
\item Le service restauration commande les produits et ingrédients mis en œuvre dans les menus
\item Le service restauration prépare les menus élaborés.
\item Chaque patient aura une quantité d'aliment correspondante au grammage de l'aliment pour sa tranche d'age (Document \ref{docNutrition}, Annexe 2).
\item La fréquence de service de chaque aliment sera conforme aux recommendations du document de nutrition \ref{docNutrition}, Annexe 4.
\item Chaque aliment devra être classé dans une des catégories d'aliments cité dans les tables de grammages et les fréquences de services du document \ref{docNutrition} Annexes 2 et 4.
\end{itemize}

\subsection{Les contraintes}
\begin{itemize}
\item Les médecins doivent pouvoir vérifier / valider les profils diététiques des patients.
\item La direction fixe un budget maximum par menu.
\end{itemize}

\subsection{Exigences}
\begin{itemize}
\item Fonctionnelles
  \begin{itemize}
  \item Chaque patient a un profil diététique, renseigné par le diététicien
  \item Chaque menu élaboré par le diététicien, correspond à un ou plusieurs profils diététiques patients.
  \item À l'issue de l'élaboration des menus, la liste des produits et
    ingrédients (avec leur quantité) est faite afin que le service
    restauration puisse les commander.
  \item La liste des différents menus à réaliser est mise à disposition du service restauration.
  \end{itemize}
\item Non fonctionnelles
  \begin{itemize}
  \item \colorbox{yellow}{À évaluer !}
  \end{itemize}
\end{itemize}

\section{\colorbox{yellow}{TODO} Estimation globale}
