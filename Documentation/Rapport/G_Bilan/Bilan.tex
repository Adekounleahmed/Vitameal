%-*- coding: utf-8 -*-
\textcolor[RGB]{46, 116, 181}{\chapter{Bilan}}

\begin{table}[!h]
\centering
\begin{tabular}{|l|c|c|c|}
\hline
& \textbf{Jean-Félix} & \textbf{Nicolas} & \textbf{Sonia} \\ \hline
Définition du problème & 40\% & 35\% & 25\% \\
Planification des activités & 45\% & 35\% & 20\% \\
Définition usine logiciel & 0\% & 100\% & 0\% \\
Rédaction rapport/présentation & 34\% & 33\% & 33\% \\
Suivi du projet (rapport d'avancement) & 70\% & 15\% & 15\% \\
Analyse des exigences & 40\% & 40\% & 20\% \\
Analyse fonctionnelle - Cas utilisation & 40\% & 40\% & 20\% \\
Architecture générale & 20\% & 80\% & 0\% \\
%Conception détaillée & & &  \\
Développement & 50\% & 50\% & 0\% \\
Modèle du domaine / MCD & 80\% & 10\% & 10\% \\
%Test d'intégration & & &  \\
Environnement techniques & 40\% & 50\% & 10\% \\ \hline
\textbf{TOTAL} & \textbf{42\%} & \textbf{44\%} & \textbf{14\%} \\ \hline
\end{tabular}
\caption{\label{Participation}Taux de participation sur le projet}
\end{table}

Même si nous ne sommes pas parvenus à le terminer, ce projet a été riche d'enseignements, et extraordinairement consommateur de temps.
Le périmètre était ambitieux, nous avons découvert plus de 90\% de ce que nous avons mis en oeuvre. Nous sommes cependant parvenus à avoir un début d'application opérationnel; vu d'où nous sommes partis, c'est plutôt bien, même si ce n'est pas satisfaisant; nous aurions préféré pouvoir le terminer.
