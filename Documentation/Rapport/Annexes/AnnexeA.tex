\chapter{Composition des repas}

\label{annexeA}

\section{Le petit-déjeuner}

\subsection{Composition}

Le petit-déjeuner comporte au minimum les éléments suivants :
\begin{itemize}
	\item une boisson : eau, jus de fruit (100\% fruit, sans sucre ajoutée), lait demi écrémé, café, café décaféinée, thé, tisane, chicorée, ... ;
	\item un aliment céréalier : pain, biscottes, ou autre produit céréalier, ... ;
	\item un produit laitier : lait, yaourt, fromage ou autre produit laitier, ... ;
	\item un fruit : fruit cru, jus de fruit, compote, purée de fruit.
\end{itemize}

Le lait est considéré comme une boisson et un produit laitier et le jus de fruit est considéré comme une boisson et comme un fruit.

Selon type de population, le petit-déjeuner peut éventuellement être complété par :
\begin{itemize}
	\item un élément lipidique  : beurre, margarine, ... ;
	\item un élément sucré : confiture, gelée, miel, ... ;
	\item un élément protidique : jambon, oeuf, ... .
\end{itemize}

\subsection{Restrictions}

Il convient d'éviter les pâtes à tartiner et les pâtisseries contenant plus de 15 \% de matières grasses, c'est à dire :

les viennoiseries (croissant, pain au chocolat, ...), les barres chocolatées, les biscuits chocolatés ou fourrés, les céréales fourrées, les beignets, les gaufres, les crêpes fourrées au chocolat, les gâteaux à la crème ou au chocolat, les brownies au chocolat et aux noix, les quatre-quarts, les gâteaux moelleux chocolatés type napolitain mini-roulé, les biscuits chocolatés, les biscuits sablés nappés de chocolat, les biscuits secs chocolatés, les galettes ou les sablés, les goûters chocolatés fourrés, les gaufrettes fourrées, les madeleines, les biscuits secs feuilletés type palmier, les cookies au chocolat.

Ainsi que les desserts suivant :

les tiramisus,les crèmes brûlées, les glaces ou les nougats glacés.

La fréquence recommandée est de 3 repas sur 20 repas successifs au maximum.

\subsection{Références}

\ref{docNutrition}, § 3.2.1 (page 17), § 4.2.1.1.4 (page 39)

\section{Le déjeuner et le souper}

\subsection{Composition}

Le déjeuner et le souper se composent de quatre ou cinq composantes selon le tableau ci-dessous. Cinq composantes donnent plus de latitude et de souplesse dans la mise en œuvre des fréquence de services (\colorbox{yellow}{TODO}).

\newcommand\Chbx{\centering{\CheckedBox}}

\begin{center}

\begin{tabular}{|l|c|c|c|c|}
	\hline
	\textbf{Composantes} & \textbf{5 composantes} & \multicolumn{3}{|c|}{\textbf{4 composantes}} \\
	\hline
	Entrée & \checkmark & \checkmark & \checkmark ** & \cellcolor{gray} \\
	\hline
	Plats protidiques & \checkmark & \checkmark & \checkmark & \checkmark \\
	\hline
	Garnitures & \checkmark & \checkmark & \checkmark & \checkmark \\
	\hline
	Produits laitiers & \checkmark & \cellcolor{gray} & \checkmark & \checkmark \\
	\hline
	Desserts & \checkmark & \checkmark ** & \checkmark & \checkmark \\
	\hline
	Pain & \multicolumn{4}{|c|}{Présence systématique} \\
	\hline
	Eau & \multicolumn{4}{|c|}{Présence systématique} \\
	\hline
\end{tabular}

\end{center}

\noindent * Seule boisson indispensable, du lait demi-écrémé non sucré peut aussi être proposé. \\
** Présence obligatoire d'un produit laitier dans l'entrée ou le dessert.

\vspace{0.5cm}

Les composantes des repas principaux sont généralement constituées de : 
\begin{itemize}
	\item Les entrées : crudités, cuidités, entrées de légumes secs et ou d’autres féculents, entrées protidiques (oeuf, poisson), préparations pâtissières salées, charcuteries ;
	\item Les plats protidiques : plat principal à base de viande, poisson, oeuf, abats. Préparations pâtissières salées servies en plat principal (crêpes salées, friands divers, pizzas, tartes, quiches, tourtes). Charcuteries servies en plat principal (préparation traditionnelle à base de chair de porc, boudin noir, saucisses diverses, crépinettes, ...) ;
	\item Les garnitures : légumes, légumes secs, pommes de terre, produits céréaliers ;
	\item Les produits laitiers : Lait demi-écrémé, lait fermenté ou autre produit laitier frais, fromage, dessert lacté ;
	\item Les desserts : fruit crus entier ou en salade, fruit cuit ou au sirop, pâtisserie, biscuit, sorbet, dessert lacté, glace.
\end{itemize}

\subsection{Restrictions}

Il est déconseillé de distribuer des boissons sucrées.

\subsection{Références}

\ref{docNutrition}, § 3.2.3 (page 18-19)