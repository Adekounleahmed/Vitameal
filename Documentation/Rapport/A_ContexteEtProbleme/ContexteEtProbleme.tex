%-*- coding: utf-8 -*-
\textcolor[RGB]{46, 116, 181}{\chapter{Contexte et problème}}
Quelle que soit l’importance des avancées scientifiques et technologiques, c’est le travail des
professionnels de santé qui détermine la qualité et l’efficacité des soins. Dans ce contexte, les soins
nutritionnels, qui portent sur l’évaluation de l’état nutritionnel et l’accompagnement alimentaire des
patients hospitalisés, en interaction étroite avec l’équipe de soin, ne font pas exception. Pour ce
faire, les diététiciens développent des actions de complexité variable, tant au niveau des services de
soins que du système de restauration.

Simultanément, les professionnels doivent faire face à de nouveaux défis, dus aux modifications des
profils épidémiologiques, démographiques et sociaux des populations, ce qui exige la mise en place
de nouvelles compétences et la reconfiguration des stratégies d’action. Pour les diététiciens du
secteur hospitalier, elles ont pour conséquences de nouvelles exigences mentales et surtout
cognitives.

Le niveau de développement industriel de la filière alimentaire française allège la charge de travail
technique des diététiciens, non seulement en ce qui concerne la diversité de matières premières,
mais également dans le domaine du contrôle \enquote{qualité}, tout au long de la chaîne de production. De
la même façon, les nouveaux concepts de production en restauration collective, caractérisés par
l’utilisation de produits pré élaborés et l’innovation technologique des équipements, gagnent
visiblement du terrain dans le secteur hospitalier français.

\section{Définition du problème}
L'élaboration de menus dans un hôpital pour la restauration des patients
est une tâche complexe, et doit tenir compte des différentes pathologies
rencontrées. Faute de moyens (temps et argent) seules quelques grandes
lignes de restauration sont retenues; alors qu'idéalement, chaque
patient devrait pourvoir avoir un repas adapté à sa pathologie.

\section{Vision du projet}
\subsection{Solution envisagée}
Le projet Vitameal a pour objectif de faire correspondre au mieux la planification des régimes et des
prescriptions diététiques aux repas réellement servis au patient. Il consiste en un outil interfaçant la
gestion de production, la prise de commande et le suivi nutritionnel des repas.

\subsection{Périmètre}
C'est un diététicien qui renseigne le profil diététique des patients,
sous les directives des médecins. C'est aussi un diététicien qui élabore
les menus des patients. L'outil élaborera donc
les menus par filtrage des produits correspondants aux profils
diététiques des patients. Pour des raisons de simplifications, nous nous limiterons dans ce projet aux seuls patients adolescents et adultes, à l'exclusion des personnes agées.
\begin{figure}[H]
\label{Modelisation_du _probleme}
  \centering
      \includegraphics[width=0.75\textwidth]{problem_model} %
\caption{Modélisation du problème}
\end{figure}

