%-*- coding: utf-8 -*-
\textcolor[RGB]{46, 116, 181}{\chapter{Analyse fonctionnelle}}

\section{Cas d'utilisation}

Les acteurs humains pour le système Vitameal sont les suivants :

Le diététicien : la personne en charge de l'élaboration des menus servis aux patients. Pour cela il doit pouvoir composer les menus des 3 repas journaliers selon les contraintes médicales de chaque patients. Il peut remplir lui-même les profils diététiques des patients mais ceux-ci doivent être validé par un médecin.

Le médecin : la personne en charge du dossier médical des patients, qui valide les profils diététiques remplit par les diététiciens.
 
Le service restauration : Les personnes en charge de la préparation et de la commande des repas.

\begin{figure}[H]
\centering
\includegraphics[scale=0.8]{../../CasDUtilisations/diagramme_cas_utilisation.png}
\caption{Diagramme des cas d'utilisation principal}
\end{figure}

Le stéréotype ``secondaire'' dans le diagramme des cas d'utilisation principal indique que le cas d'utilisation ne fait pas partie des cas d'utilisation principaux et qu'il n'est pas obligatoire pour que le système fonctionne.

\section{Description des cas d'utilisation}

\input{../CasDUtilisations/PreparerMenus/uc_preparer_menus.tex}

\input{../CasDUtilisations/ConsulterMenus/uc_consulter_menus.tex}

\input{../CasDUtilisations/CompositionPlat/uc_composer_plat.tex}

%-*- coding: utf-8 -*-
\subsection{MenuGen}
\begin{figure}
  \centering
      \includegraphics[width=1.00\textwidth]{../../CasDUtilisations/MenuGen/Sequence/EMSeq.png} %
\caption{Séquence détaillée d'élaboration des menus}
\label{MenuGenSeqDetail}
\end{figure}

\textbf{Algorithme (voir \autoref{ClassesMenuGen}):}
\begin{enumerate}
\item Sélectionner le groupe de patient dont il faut élaborer les menus.
\item Extraire de ce groupe de patients une table des contraintes (\emph{Formes}, \emph{Familles}, \emph{Textures}, \emph{AlimentsBase}) voir \autoref{ModeleDuDomaine}.
\item Générer une table contenant en plus des identifiants de plats la liste de leurs ingrédients (mêmes attributs que les contraintes).
\item Retirer de cette table, les plats incompatibles avec les contraintes listées dans la première table.
\item En comparant la table de plats résultante avec les menus déjà pris, extraire de la liste des plats (première table) les plats compatibles avec les fréquences de services.
\item élaborer les menus à partir de la liste des plats restants.
\end{enumerate}

\begin{figure}
  \centering
      \includegraphics[width=1.00\textwidth]{../../CasDUtilisations/MenuGen/Classes/EMC.png} %
\caption{classes d'élaboration des menus}
\label{ClassesMenuGen}
\end{figure}


\input{../CasDUtilisations/ProfilPatient/uc_renseigner_profil_patient}

\input{../backlog/backlog.tex}
