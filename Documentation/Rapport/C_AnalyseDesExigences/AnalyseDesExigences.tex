%-*- coding: utf-8 -*-
\textcolor[RGB]{46, 116, 181}{\chapter{Analyse des exigences}}
\section{Partie prenantes}
\begin{itemize}
\item Participantes~: les diététiciens, le service restauration
\item Concernés~: les médecins, la direction (budget)
\item Impactées~: les patients
\end{itemize}

\section{Les besoins}
\begin{description}
\item[] En tant que diététicien, j’ai besoin de :
\begin{description}
 \item[N001:] pouvoir renseigner le profil diététique des patients, afin qu’ils
 puissent bénéficier de menus adaptés.
 \item[N002:] pouvoir élaborer les menus des 3 repas journaliers
 (petit-déjeuner, déjeuner et dîner dont la composition est décrite en annexe
 \ref{annexeA}), de façon automatique, en tenant compte de grammages dépendant du type d'aliments et de la tranche d'age (Document \ref{docNutrition}, Annexe 2).
 \item[N003:] pouvoir saisir des plats et leur composition.
 \item[N004:] élaborer des menus selon les fréquences de service, selon le
 document \ref{docNutrition}, Annexe 4.
 \item[N005:] classer chaque aliment dans une des catégories d’aliments citée
 dans les tables de grammages du document \ref{docNutrition}, Annexes 2 et 4.
\end{description}
\item[N006:] En tant qu’administrateur du site internet de l’hôpital, j’ai
besoin de récupérer le menu de la semaine, afin pouvoir l’intégrer au site.
\item[N007:] En tant que médecin, j’ai besoin de consulter les profils
diététiques des patients admis, pour les  valider.
\item[N008:] En tant que cuisinier du service restauration, j’ai besoin de
consulter les menus élaborés, afin de pouvoir les préparer et prévoir les ingrédients à commander.
\item[N009:] En tant qu’agent de restauration hospitalière,  j’ai besoin de
connaître les menus de chaque patient, afin de pouvoir assembler les plateaux repas.
\end{description}
\section{Les contraintes}
\begin{description}
\item[N010:] Les médecins doivent pouvoir vérifier / valider les profils
diététiques des patients.
\item[N011:] La direction fixe un budget maximum par menu.
\end{description}

\section{Exigences}

\rowcolors{1}{}{}

\begin{table}[!h]

\begin{tabular}{|p{60mm}p{100mm}|}

\hline

\multicolumn{2}{|l|}{\textbf{REQ\_0100:} 3 repas} \\ \hline

\emph{Type:} Métier & \emph{Liens:} REQ\_0101 REQ\_0105  \\

\emph{Origine:}  & \emph{Validé:} Non \\

\emph{Version:} Initial & \emph{Test:}  \\

\emph{Priorité:} Must & \\ \hline

\multicolumn{2}{|p{16cm}|}{Le système doit permettre de concevoir les 3 repas (petit-déjeuner, déjeuner, souper) d'une journée.} \\ \hline

\end{tabular}

\end{table}



\begin{table}[!h]

\begin{tabular}{|p{60mm}p{100mm}|}

\hline

\multicolumn{2}{|l|}{\textbf{REQ\_0101:} Petit déjeuner} \\ \hline

\emph{Type:} Métier & \emph{Liens:} REQ\_0102 REQ\_0103 REQ\_0104  \\

\emph{Origine:}  & \emph{Validé:} Non \\

\emph{Version:} Initial & \emph{Test:}  \\

\emph{Priorité:} Should & \\ \hline

\multicolumn{2}{|p{16cm}|}{Le système doit permettre de concevoir un petit-déjeuner composé d'une boisson, d'un aliment céréalier, d'un produit laitier et d'un fruit.} \\ \hline

\end{tabular}

\end{table}



\begin{table}[!h]

\begin{tabular}{|p{60mm}p{100mm}|}

\hline

\multicolumn{2}{|l|}{\textbf{REQ\_0102:} Éléments petit déjeuner} \\ \hline

\emph{Type:} Métier & \emph{Liens:}  \\

\emph{Origine:}  & \emph{Validé:} Non \\

\emph{Version:} Initial & \emph{Test:}  \\

\emph{Priorité:} Must & \\ \hline

\multicolumn{2}{|p{16cm}|}{Le système doit permettre de rajouter au petit déjeuner un élément lipidique, sucré ou protodique.} \\ \hline

\end{tabular}

\end{table}



\begin{table}[!h]

\begin{tabular}{|p{60mm}p{100mm}|}

\hline

\multicolumn{2}{|l|}{\textbf{REQ\_0103:} Éléments non diététiques} \\ \hline

\emph{Type:} Métier & \emph{Liens:}  \\

\emph{Origine:}  & \emph{Validé:} Non \\

\emph{Version:} Initial & \emph{Test:}  \\

\emph{Priorité:} Should & \\ \hline

\multicolumn{2}{|p{16cm}|}{Le système doit avertir l'utilisateur de l'usage d'élément non diététique dans un petit déjeuner.} \\ \hline

\end{tabular}

\end{table}



\begin{table}[!h]

\begin{tabular}{|p{60mm}p{100mm}|}

\hline

\multicolumn{2}{|l|}{\textbf{REQ\_0104:} Fréquence éléments non diététiques} \\ \hline

\emph{Type:} Métier & \emph{Liens:}  \\

\emph{Origine:}  & \emph{Validé:} Non \\

\emph{Version:} Initial & \emph{Test:}  \\

\emph{Priorité:} Should & \\ \hline

\multicolumn{2}{|p{16cm}|}{Le système doit vérifier que la fréquence de l'usage d'élément non diététique des petits déjeuners ne dépasse pas 3 repas sur 20, il avertit l'utilisateur si c'est le cas.} \\ \hline

\end{tabular}

\end{table}



\begin{table}[!h]

\begin{tabular}{|p{60mm}p{100mm}|}

\hline

\multicolumn{2}{|l|}{\textbf{REQ\_0105:} Composition déjeuner} \\ \hline

\emph{Type:} Métier & \emph{Liens:}  \\

\emph{Origine:}  & \emph{Validé:} Non \\

\emph{Version:} Initial & \emph{Test:}  \\

\emph{Priorité:} Must & \\ \hline

\multicolumn{2}{|p{16cm}|}{Le système de concevoir un déjeuner et souper composés de 4 ou cinq composantes parmi : entrée, plat protodique, garniture, produit, laitier desserts + de l'eau et du pain (selon le tableau sur la composition du déjeuner en annexe A).} \\ \hline

\end{tabular}

\end{table}



\begin{table}[!h]

\begin{tabular}{|p{60mm}p{100mm}|}

\hline

\multicolumn{2}{|l|}{\textbf{REQ\_0106:} Ajout de plats} \\ \hline

\emph{Type:} Métier & \emph{Liens:}  \\

\emph{Origine:}  & \emph{Validé:} Non \\

\emph{Version:} Initial & \emph{Test:}  \\

\emph{Priorité:} Must & \\ \hline

\multicolumn{2}{|p{16cm}|}{Le système doit permettre d'ajouter des plats et leur définition dans la listes des plats pouvant être préparés.} \\ \hline

\end{tabular}

\end{table}



\begin{table}[!h]

\begin{tabular}{|p{60mm}p{100mm}|}

\hline

\multicolumn{2}{|l|}{\textbf{REQ\_0107:} Description d'un plat} \\ \hline

\emph{Type:} Métier & \emph{Liens:}  \\

\emph{Origine:}  & \emph{Validé:} Non \\

\emph{Version:} Initial & \emph{Test:}  \\

\emph{Priorité:} Must & \\ \hline

\multicolumn{2}{|p{16cm}|}{Le système doit permettre la description d'un plat avec sa liste d'ingrédients et les quantités nécessaires à sa réalisation.} \\ \hline

\end{tabular}

\end{table}



\begin{table}[!h]

\begin{tabular}{|p{60mm}p{100mm}|}

\hline

\multicolumn{2}{|l|}{\textbf{REQ\_0108:} Fréquence de service} \\ \hline

\emph{Type:} Métier & \emph{Liens:}  \\

\emph{Origine:}  & \emph{Validé:} Non \\

\emph{Version:} Initial & \emph{Test:}  \\

\emph{Priorité:} Must & \\ \hline

\multicolumn{2}{|p{16cm}|}{Le système doit proposer un plat selon la fréquence de service de ce plat (exemple 4 fois tous les 20 repas).} \\ \hline

\end{tabular}

\end{table}



\begin{table}[!h]

\begin{tabular}{|p{60mm}p{100mm}|}

\hline

\multicolumn{2}{|l|}{\textbf{REQ\_0410:} Composants des repas} \\ \hline

\emph{Type:} Métier & \emph{Liens:}  \\

\emph{Origine:}  & \emph{Validé:} Non \\

\emph{Version:} Initial & \emph{Test:}  \\

\emph{Priorité:} Must & \\ \hline

\multicolumn{2}{|p{16cm}|}{Le système doit permettre d'ajouter et de supprimer des éléments dans les composants des repas.} \\ \hline

\end{tabular}

\end{table}



\begin{table}[!h]

\begin{tabular}{|p{60mm}p{100mm}|}

\hline

\multicolumn{2}{|l|}{\textbf{REQ\_0411:} Listes par défaut} \\ \hline

\emph{Type:} Métier & \emph{Liens:}  \\

\emph{Origine:}  & \emph{Validé:} Non \\

\emph{Version:} Initial & \emph{Test:}  \\

\emph{Priorité:} Should & \\ \hline

\multicolumn{2}{|p{16cm}|}{Le système doit permettre de revenir aux listes par défaut recommandé par le gouvernement.} \\ \hline

\end{tabular}

\end{table}



\begin{table}[!h]

\begin{tabular}{|p{60mm}p{100mm}|}

\hline

\multicolumn{2}{|l|}{\textbf{REQ\_0500:} Fiche de commande} \\ \hline

\emph{Type:} Métier & \emph{Liens:}  \\

\emph{Origine:}  & \emph{Validé:} Non \\

\emph{Version:} Initial & \emph{Test:}  \\

\emph{Priorité:} Could & \\ \hline

\multicolumn{2}{|p{16cm}|}{Le système doit permettre, une fois les menus élaborés de générer un fiche de commande au format : à définir.} \\ \hline

\end{tabular}

\end{table}



\begin{table}[!h]

\begin{tabular}{|p{60mm}p{100mm}|}

\hline

\multicolumn{2}{|l|}{\textbf{REQ\_0501:} Publication menus} \\ \hline

\emph{Type:} Non Fonctionnelle & \emph{Liens:}  \\

\emph{Origine:}  & \emph{Validé:} Non \\

\emph{Version:} Initial & \emph{Test:}  \\

\emph{Priorité:} Could & \\ \hline

\multicolumn{2}{|p{16cm}|}{Le système doit permettre d'afficher les menus sur un site internet.} \\ \hline

\end{tabular}

\end{table}



\begin{table}[!h]

\begin{tabular}{|p{60mm}p{100mm}|}

\hline

\multicolumn{2}{|l|}{\textbf{REQ\_0600:} Validation des repas} \\ \hline

\emph{Type:} Contrainte & \emph{Liens:}  \\

\emph{Origine:}  & \emph{Validé:} Non \\

\emph{Version:} Initial & \emph{Test:}  \\

\emph{Priorité:} Must & \\ \hline

\multicolumn{2}{|p{16cm}|}{Le système doit gérer un cycle de validation des repas : en cours d'élaboration, en attente de validation, validé.} \\ \hline

\end{tabular}

\end{table}



\begin{table}[!h]

\begin{tabular}{|p{60mm}p{100mm}|}

\hline

\multicolumn{2}{|l|}{\textbf{REQ\_0601:} Droits utilisateurs} \\ \hline

\emph{Type:} Contrainte & \emph{Liens:}  \\

\emph{Origine:}  & \emph{Validé:} Non \\

\emph{Version:} Initial & \emph{Test:}  \\

\emph{Priorité:} Must & \\ \hline

\multicolumn{2}{|p{16cm}|}{Le système doit permettre de gérer différent droit selon le type d'utilisateur.} \\ \hline

\end{tabular}

\end{table}



\begin{table}[!h]

\begin{tabular}{|p{60mm}p{100mm}|}

\hline

\multicolumn{2}{|l|}{\textbf{REQ\_0700:} Menus à assembler} \\ \hline

\emph{Type:} Métier & \emph{Liens:}  \\

\emph{Origine:}  & \emph{Validé:} Non \\

\emph{Version:} Initial & \emph{Test:}  \\

\emph{Priorité:} Must & \\ \hline

\multicolumn{2}{|p{16cm}|}{Le système doit afficher les menu à assembler pour un jour donnée et émettre une étiquette au format : à définir.} \\ \hline

\end{tabular}

\end{table}



\begin{table}[!h]

\begin{tabular}{|p{60mm}p{100mm}|}

\hline

\multicolumn{2}{|l|}{\textbf{REQ\_0701:} Limite prix repas} \\ \hline

\emph{Type:} Contrainte & \emph{Liens:}  \\

\emph{Origine:}  & \emph{Validé:} Non \\

\emph{Version:} Initial & \emph{Test:}  \\

\emph{Priorité:} Must & \\ \hline

\multicolumn{2}{|p{16cm}|}{Le système doit permettre de fixer une limite au prix d'un repas.} \\ \hline

\end{tabular}

\end{table}



\begin{table}[!h]

\begin{tabular}{|p{60mm}p{100mm}|}

\hline

\multicolumn{2}{|l|}{\textbf{REQ\_0702:} Prix repas} \\ \hline

\emph{Type:} Métier & \emph{Liens:}  \\

\emph{Origine:}  & \emph{Validé:} Non \\

\emph{Version:} Initial & \emph{Test:}  \\

\emph{Priorité:} Must & \\ \hline

\multicolumn{2}{|p{16cm}|}{Le système doit permettre de renseigner le prix des éléments d'un repas.} \\ \hline

\end{tabular}

\end{table}



\begin{table}[!h]

\begin{tabular}{|p{60mm}p{100mm}|}

\hline

\multicolumn{2}{|l|}{\textbf{REQ\_0902:} Profil patient} \\ \hline

\emph{Type:} Métier & \emph{Liens:}  \\

\emph{Origine:}  & \emph{Validé:} Non \\

\emph{Version:} Initial & \emph{Test:}  \\

\emph{Priorité:} Must & \\ \hline

\multicolumn{2}{|p{16cm}|}{Le système doit permettre de renseigner un profil patient comportant les éléments suivants : regime particulier(liste à définir), allergie(liste à définir), contre-indication (liste à définir).} \\ \hline

\end{tabular}

\end{table}



\begin{table}[!h]

\begin{tabular}{|p{60mm}p{100mm}|}

\hline

\multicolumn{2}{|l|}{\textbf{REQ\_1000:} État civil} \\ \hline

\emph{Type:} Métier & \emph{Liens:}  \\

\emph{Origine:}  & \emph{Validé:} Non \\

\emph{Version:} Initial & \emph{Test:}  \\

\emph{Priorité:} Must & \\ \hline

\multicolumn{2}{|p{16cm}|}{Le système doit permettre de renseigner l'état civil d'un patient.} \\ \hline

\end{tabular}

\end{table}



\begin{table}[!h]

\begin{tabular}{|p{60mm}p{100mm}|}

\hline

\multicolumn{2}{|l|}{\textbf{REQ\_1001:} Localisation patient} \\ \hline

\emph{Type:} Métier & \emph{Liens:}  \\

\emph{Origine:}  & \emph{Validé:} Non \\

\emph{Version:} Initial & \emph{Test:}  \\

\emph{Priorité:} Must & \\ \hline

\multicolumn{2}{|p{16cm}|}{Le système doit permettre de renseigner la localisation particulière d'un patient.} \\ \hline

\end{tabular}

\end{table}



\begin{table}[!h]

\begin{tabular}{|p{60mm}p{100mm}|}

\hline

\multicolumn{2}{|l|}{\textbf{REQ\_1002:} Grammages} \\ \hline

\emph{Type:} Métier & \emph{Liens:}  \\

\emph{Origine:}  & \emph{Validé:} Non \\

\emph{Version:} Initial & \emph{Test:}  \\

\emph{Priorité:} Must & \\ \hline

\multicolumn{2}{|p{16cm}|}{Le système doit permettre de gérer les grammage de plat.} \\ \hline

\end{tabular}

\end{table}



\begin{table}[!h]

\begin{tabular}{|p{60mm}p{100mm}|}

\hline

\multicolumn{2}{|l|}{\textbf{REQ\_1003:} Plateaux repas} \\ \hline

\emph{Type:} Métier & \emph{Liens:}  \\

\emph{Origine:}  & \emph{Validé:} Non \\

\emph{Version:} Initial & \emph{Test:}  \\

\emph{Priorité:} Must & \\ \hline

\multicolumn{2}{|p{16cm}|}{Le système doit pouvoir gérer des plateaux repas de type : sans régime particulier ou avec régime particulier.} \\ \hline

\end{tabular}

\end{table}



\begin{table}[!h]

\begin{tabular}{|p{60mm}p{100mm}|}

\hline

\multicolumn{2}{|l|}{\textbf{REQ\_1004:} Groupes} \\ \hline

\emph{Type:} Métier & \emph{Liens:}  \\

\emph{Origine:}  & \emph{Validé:} Non \\

\emph{Version:} Initial & \emph{Test:}  \\

\emph{Priorité:} Should & \\ \hline

\multicolumn{2}{|p{16cm}|}{Le système doit gérer les patients par groupes selon leur régime, exemple le groupe des intolérant au lactose.} \\ \hline

\end{tabular}

\end{table}



\begin{table}[!h]

\begin{tabular}{|p{60mm}p{100mm}|}

\hline

\multicolumn{2}{|l|}{\textbf{REQ\_1005:} Génération automatique} \\ \hline

\emph{Type:} Fonctionnelle & \emph{Liens:}  \\

\emph{Origine:}  & \emph{Validé:} Non \\

\emph{Version:} Initial & \emph{Test:}  \\

\emph{Priorité:} Must & \\ \hline

\multicolumn{2}{|p{16cm}|}{Le système doit permettre de générer automatiquement les repas pour un groupe de patients particulier.} \\ \hline

\end{tabular}

\end{table}



\begin{table}[!h]

\begin{tabular}{|p{60mm}p{100mm}|}

\hline

\multicolumn{2}{|l|}{\textbf{REQ\_1006:} Titre} \\ \hline

\emph{Type:} Utilisateur & \emph{Liens:}  \\

\emph{Origine:} Origine & \emph{Validé:} Oui \\

\emph{Version:} Initial & \emph{Test:} Test \\

\emph{Priorité:} Must & \\ \hline

\multicolumn{2}{|p{16cm}|}{le système doit stocker les fiches patients, et permettre de les modifier ou supprimer le cas échéant.} \\ \hline

\end{tabular}

\end{table}



\begin{table}[!h]

\begin{tabular}{|p{60mm}p{100mm}|}

\hline

\multicolumn{2}{|l|}{\textbf{REQ\_1007:} Titre} \\ \hline

\emph{Type:} Utilisateur & \emph{Liens:}  \\

\emph{Origine:} Origine & \emph{Validé:} Oui \\

\emph{Version:} Initial & \emph{Test:} Test \\

\emph{Priorité:} Must & \\ \hline

\multicolumn{2}{|p{16cm}|}{Le système doit permettre de trier les plats par catégories.} \\ \hline

\end{tabular}

\end{table}



\begin{table}[!h]

\begin{tabular}{|p{60mm}p{100mm}|}

\hline

\multicolumn{2}{|l|}{\textbf{REQ\_1008:} Titre} \\ \hline

\emph{Type:} Utilisateur & \emph{Liens:}  \\

\emph{Origine:} Origine & \emph{Validé:} Oui \\

\emph{Version:} Initial & \emph{Test:} Test \\

\emph{Priorité:} Must & \\ \hline

\multicolumn{2}{|p{16cm}|}{Le système doit stocker les intitulés des plats, et permettre leur modification ou leur suppression.} \\ \hline

\end{tabular}

\end{table}





