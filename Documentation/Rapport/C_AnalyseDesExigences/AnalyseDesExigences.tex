%-*- coding: utf-8 -*-
\textcolor[RGB]{46, 116, 181}{\chapter{Analyse des exigences}}
\section{Partie prenantes}
\begin{itemize}
\item Participantes~: les diététiciens, le service restauration
\item Concernés~: les médecins, la direction (budget)
\item Impactées~: les patients
\end{itemize}

\section{Les besoins}
\begin{description}
\item[] En tant que diététicien, j’ai besoin de :
\begin{description}
 \item[N001:] pouvoir renseigner le profil diététique des patients, afin qu’ils
 puissent bénéficier de menus adaptés.
 \item[N002:] pouvoir élaborer les menus des 3 repas journaliers
 (petit-déjeuner, déjeuner et dîner dont la composition est décrite en annexe
 \ref{annexeA}), de façon automatique, en tenant compte de grammages dépendant du type d'aliments et de la tranche d'age (Document \ref{docNutrition}, Annexe 2).
 \item[N003:] pouvoir saisir des plats et leur composition.
 \item[N004:] élaborer des menus selon les fréquences de service, selon le
 document \ref{docNutrition}, Annexe 4.
 \item[N005:] classer chaque aliment dans une des catégories d’aliments citée
 dans les tables de grammages du document \ref{docNutrition}, Annexes 2 et 4.
\end{description}
\item[N006:] En tant qu’administrateur du site internet de l’hôpital, j’ai
besoin de récupérer le menu de la semaine, afin pouvoir l’intégrer au site.
\item[N007:] En tant que médecin, j’ai besoin de consulter les profils
diététiques des patients admis, pour les  valider.
\item[N008:] En tant que cuisinier du service restauration, j’ai besoin de
consulter les menus élaborés, afin de pouvoir les préparer et prévoir les ingrédients à commander.
\item[N009:] En tant qu’agent de restauration hospitalière,  j’ai besoin de
connaître les menus de chaque patient, afin de pouvoir assembler les plateaux repas.
\end{description}
\section{Les contraintes}
\begin{description}
\item[N010:] Les médecins doivent pouvoir vérifier / valider les profils
diététiques des patients.
\item[N011:] La direction fixe un budget maximum par menu.
\end{description}

\section{Exigences}

\input{../Exigences/Exigences}

