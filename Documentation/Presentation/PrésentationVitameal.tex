%-*- coding: utf-8 -*-
\documentclass{beamer}

\usepackage[frenchb]{babel}
\usepackage[T1]{fontenc}
\usepackage[utf8]{inputenc}
\graphicspath{{images/}}
\usepackage{csquotes}
\usepackage{tocvsec2}
\maxtocdepth{section}
\usetheme{Warsaw}
\setbeamertemplate{headline}{} % Supprime la zone de menu en haut (prend trop de place !)

\title{Projet Vitameal}
\subtitle{Restauration en milieu hospitalier}
\author{Nicolas Symphorien, Sonia Othmani, Jean-Félix Benitez} % Si on met les noms en majuscules, il y en a un de tronqué en bas dans le pied de page.
\institute{CNAM}
\date{13/09/2017}

\logo{\includegraphics[height=10mm]{ipst-cnam.png}}
\setbeamertemplate{background canvas}{\includegraphics[width=\paperwidth,height=\paperheight]{fond_200.png}} % Width pour la largeur, height pour la hauteur de l'image

\begin{document}
\begin{frame}[plain]
  \titlepage
\end{frame}

\begin{frame}
  \frametitle{Sommaire}
  \tableofcontents
\end{frame}

\section{Contexte et problème}
\begin{frame}[label=definitionDuProbleme]
  \frametitle{Définition du problème}
  \rightskip=0pt\leftskip=0pt
L'élaboration de menus dans un hôpital pour la restauration des patients
est une tâche complexe, et doit tenir compte des différentes pathologies
rencontrées. Faute de moyens (temps et argent) seules quelques grandes
lignes de restauration sont retenues; alors qu'idéalement, chaque
patient devrait pourvoir avoir un repas adapté à sa pathologie.
\end{frame}

%\subsection{Solution envisagée}
%\begin{frame}[label=solutionEnvisagée]
%\frametitle{Solution envisagée}
%Le projet Vitameal a pour objectif de faire correspondre au mieux la planification des régimes et des
%prescriptions diététiques aux repas réellement servis au patient. Il consiste en un outil interfaçant la
%gestion de production, la prise de commande et le suivi nutritionnel des repas.
%\end{frame}

\subsection{Périmètre}
\begin{frame}[label=perimetre]
  \frametitle{Périmètre}
  \rightskip=0pt\leftskip=0pt
C'est un diététicien qui renseigne le profil diététique des patients,
sous les directives des médecins. C'est aussi un diététicien qui élabore
les menus des patients. L'outil élaborera donc
les menus par filtrage des produits correspondants aux profils
diététiques des patients. Pour des raisons de simplifications, nous nous limiterons dans ce projet aux seuls patients adolescents et adultes, à l'exclusion des personnes agées.
\end{frame}

\section{Méthode de travail}
%\subsection{Organisation}
%\begin{frame}[label=organisation]
%\frametitle{Organisation}
%\end{frame}

\subsection{Usine logicielle}
\begin{frame}[label=schemaFonctionnement]
  \frametitle{Usine logicielle - Schéma de fonctionnement}
\begin{figure}[H]
\label{schema}
  \centering
      \includegraphics[width=1.0\textwidth]{usine_vitameal.png} %
%\caption{Schéma}
\end{figure}
\end{frame}

\section{Analyse des exigences}
\subsection{Analyse}
\begin{frame}[label=analyseDesExigences] %allowframebreaks
\frametitle{Analyse des exigences}
\begin{itemize}
  \item Partie prenantes
  \begin{itemize}
    \item Participantes~: les diététiciens, le service restauration
    \item Concernés~: les médecins, la direction (budget)
    \item Impactées~: les patients
  \end{itemize}
  \item Les besoins
  \begin{itemize}
    \item Les diététiciens renseignent les profils diététiques de chaque patient.
    \item Les diététiciens lance l'élaboration automatique des menus.
    \item Le service restauration commande les produits et ingrédients mis en œuvre dans les menus
    \item Le service restauration prépare les menus élaborés.
  \end{itemize}
  \item Les contraintes
  \begin{itemize}
    \item Les médecins doivent pouvoir vérifier / valider les profils diététiques des patients.
    \item La direction fixe un budget maximum par menu.
  \end{itemize}
\end{itemize}
\end{frame}

\subsection{Exigences}
\begin{frame}
 \frametitle{Exigences}
Chaque exigence est composée de 11 champs:
\begin{itemize}
\item \textbf{numéro:} Formé comme suit REQ\_12345
\item \textbf{Titre:} Titre ou description courte
\item \textbf{Corps:} Expression de l'exigence
\item \textbf{Type:} Utilisateur, Métier, Système, Contrainte
\item \textbf{Nature:} Fonctionnelle, Ergonomie, Robustesse, Performance, Sécurité
\item \textbf{Origine:} D'où vient une exigence ?
\item \textbf{Version:} ou niveau de maturité, Initiale, Intermédiaire, Finale
\item \textbf{Priorité:} MoSCoW, Must, Should, Could, Won't
\item \textbf{Validée:} L'exigences a-t-elle été validée ? (Oui / Non)
\item \textbf{Liens:} Liens
\item \textbf{Test:} Définition du test qui validera l'exigence.
\end{itemize}

\url{../Exigences/Exigences.html}
\end{frame}

\section{Analyse fonctionnelle}

\begin{frame}
\frametitle{Cas d'Utilisations}
\begin{figure}[H]
\label{schema}
  \centering
      \includegraphics[scale=0.4]{../CasDUtilisations/diagramme_cas_utilisation.png}
\end{figure}
\end{frame}

\subsection{Profil patient}
\begin{frame}[plain]{}
%\frametitle{Profil patient}
%\input{../CasDUtilisations/ProfilPatient/uc_renseigner_profil_patient.tex}
\begin{figure}
\centering
\includegraphics[scale=0.4]{../CasDUtilisations/ProfilPatient/UseCaseProfilPatient.png}
\end{figure}
\end{frame}

\begin{frame}[plain]{}
\begin{figure}
\centering
\includegraphics[scale=0.4]{../CasDUtilisations/ProfilPatient/diagseqProfilPatient.png}
\end{figure}
\end{frame}

\begin{frame}[plain]{}
\begin{figure}
\centering
\includegraphics[scale=0.4]{../CasDUtilisations/ProfilPatient/diagSrqDetaillProfilPatient.png}
\end{figure}
\end{frame}

\begin{frame}[plain]{}
\begin{figure}
\centering
\includegraphics[scale=0.5]{../CasDUtilisations/ProfilPatient/diagclassProfilPatient.png}
\end{figure}
\end{frame}

\subsection{Composer un plat}

\begin{frame}[plain]{}
\begin{figure}
\centering
\includegraphics[scale=0.125]{../CasDUtilisations/CompositionPlat/sequence_UC_ComposerPlat.png}
\end{figure}
\end{frame}

\begin{frame}[plain]{}
\begin{figure}
\centering
\includegraphics[scale=0.4]{../CasDUtilisations/CompositionPlat/maquette_EcranConsulterPlats.png}
\caption{Maquette de consultation d'un plat}
\label{MaquetteConsultationPlat}
\end{figure}
\end{frame}

\begin{frame}[plain]{}
\begin{figure}
\centering
\includegraphics[scale=0.4]{../CasDUtilisations/CompositionPlat/maquette_EcranCreationPlat.png}
\caption{Maquette de la création d'un plat}
\label{MaquetteCreationPlat}
\end{figure}
\end{frame}

\begin{frame}[plain]{}
\begin{figure}
\centering
\includegraphics[scale=0.4]{../CasDUtilisations/CompositionPlat/maquette_EcranEditionPlat.png}
\caption{Maquette de l'édition d'un plat}
\label{MaquetteEditionPlat}
\end{figure}
\end{frame}

\begin{frame}[plain]{}
\begin{figure}
\centering
\includegraphics[scale=0.4]{../CasDUtilisations/CompositionPlat/maquette_MessageSupressionPlat.png}
\caption{Maquette de suppression d'un plat}
\label{MaquetteSuppressionPlat}
\end{figure}
\end{frame}

\begin{frame}[plain]{}
\begin{figure}
\centering
\includegraphics[scale=0.325]{../CasDUtilisations/CompositionPlat/sequence_InitialisationPlatControleur.png}
\end{figure}
\end{frame}

\begin{frame}[plain]{}
\begin{figure}
\centering
\includegraphics[scale=0.325]{../CasDUtilisations/CompositionPlat/sequence_CreerPlat.png}
\end{figure}
\end{frame}

\begin{frame}[plain]{}
\begin{figure}
\centering
\includegraphics[scale=0.275]{../CasDUtilisations/CompositionPlat/sequence_EditerPlat.png}
\end{figure}
\end{frame}

\begin{frame}[plain]{}
\begin{figure}
\centering
\includegraphics[scale=0.325]{../../CasDUtilisations/CompositionPlat/sequence_SupprimerPlat.png}
\end{figure}
\end{frame}

\begin{frame}[plain]{}
\begin{figure}
\centering
\includegraphics[scale=0.175]{../CasDUtilisations/CompositionPlat/classDiagram_ComposerPlat.png}
\end{figure}
\end{frame}

\subsection{Génération des menus}
\begin{frame}[plain]{}
%\frametitle{Génération des menus}
%%-*- coding: utf-8 -*-
\subsection{MenuGen}
\begin{figure}
  \centering
      \includegraphics[width=1.00\textwidth]{../../CasDUtilisations/MenuGen/Sequence/EMSeq.png} %
\caption{Séquence détaillée d'élaboration des menus}
\label{MenuGenSeqDetail}
\end{figure}

\textbf{Algorithme (voir \autoref{ClassesMenuGen}):}
\begin{enumerate}
\item Sélectionner le groupe de patient dont il faut élaborer les menus.
\item Extraire de ce groupe de patients une table des contraintes (\emph{Formes}, \emph{Familles}, \emph{Textures}, \emph{AlimentsBase}) voir \autoref{ModeleDuDomaine}.
\item Générer une table contenant en plus des identifiants de plats la liste de leurs ingrédients (mêmes attributs que les contraintes).
\item Retirer de cette table, les plats incompatibles avec les contraintes listées dans la première table.
\item En comparant la table de plats résultante avec les menus déjà pris, extraire de la liste des plats (première table) les plats compatibles avec les fréquences de services.
\item élaborer les menus à partir de la liste des plats restants.
\end{enumerate}

\begin{figure}
  \centering
      \includegraphics[width=1.00\textwidth]{../../CasDUtilisations/MenuGen/Classes/EMC.png} %
\caption{classes d'élaboration des menus}
\label{ClassesMenuGen}
\end{figure}

\begin{figure}
\centering
\includegraphics[scale=.500]{../CasDUtilisations/MenuGen/CasDUtilisation/MenuGen.png}
\end{figure}
\end{frame}

\begin{frame}[plain]{}
\begin{figure}
\centering
\includegraphics[scale=0.080]{../CasDUtilisations/MenuGen/Sequence/ElaborationMenus.png}
\end{figure}
\end{frame}

\begin{frame}[plain]{}
\begin{figure}
\centering
\includegraphics[scale=0.350]{../CasDUtilisations/MenuGen/Classes/EMC.png}
\end{figure}
\end{frame}

\subsection{Affichage des menus}
\begin{frame}
\frametitle{Affichage des menus}
\input{../CasDUtilisations/AfficherMenu/uc_afficher_menu.tex}
\end{frame}

\section{Architecture et choix techniques}

\begin{frame}[label=Architecture 3-tiers]
  \frametitle{Architecture 3-tiers}
\begin{figure}[H]
\label{schema}
  \centering
      \includegraphics[width=1\textwidth]{architectureVitameal_3tiers.png} %
%\caption{Schéma}
\end{figure}
\end{frame}

\begin{frame}[label=Détails de l'architecture]
  \frametitle{Détails de l'architecture}
\begin{figure}[H]
\label{schema}
  \centering
      \includegraphics[width=0.35\textwidth]{architectureVitameal_details.png} %
%\caption{Schéma}
\end{figure}
\end{frame}

\section{Codage}
\subsection{Modèle du domaine}
\begin{frame}[label=MDD]
  \frametitle{Modèle du domaine}
  \begin{figure}
    \centering
    \includegraphics[scale=0.12]{../ModeleDuDomaine/ModeleDuDomaine.png}
  \end{figure}
\end{frame}

\section{Bilan}
\begin{frame}[label=Bilan]
  \frametitle{Bilan}
  \rightskip=0pt\leftskip=0pt
  Même si nous ne sommes pas parvenus à le terminer, ce projet a été riche d'enseignements, et extraordinairement consommateur de temps.
Le périmètre était ambitieux, nous avons découvert plus de 90\% de ce que nous avons mis en oeuvre. Nous sommes cependant parvenus à avoir un début d'application opérationnel; vu d'où nous sommes partis, c'est plutôt bien, même si ce n'est pas satisfaisant; nous aurions préféré pouvoir le terminer.

\end{frame}

\end{document}
