%-*- coding: utf-8 -*-
\documentclass[11pt,a4paper,french,twoside,openright]{article}
\usepackage[utf8]{inputenc}
\usepackage[T1]{fontenc}
\usepackage{graphicx}%pour insérer images et pdf entre autres
	\graphicspath{{images/}}%pour spécifier le chemin d'accès aux images
\usepackage[left=2.5cm,right=2.5cm,top=2.5cm,bottom=2.5cm]{geometry}%réglages des marges du document selon vos préférences ou celles de votre établissemant
\usepackage[Lenny]{fncychap}%pour de jolis titres de chapitres voir la doc pour d'autres styles.
\usepackage{babel}
\usepackage[babel=true]{csquotes} % csquotes va utiliser la langue définie dans babel

\usepackage{fancyhdr}%pour les entêtes et pieds de pages
	\setlength{\headheight}{14.2pt}% hauteur de l'entête
        \chead{\textbf{VITAMEAL}}
        \lhead{}
        \rhead{}
	\cfoot{Formation Ingénieur Informatique en alternance - Première année}
	\lfoot{\textbf{CNAM}}%
  	\rfoot{\textbf{\thepage/\pageref{LastPage}}}
 	\renewcommand{\headrulewidth}{0.4pt}%trait horizontal pour l'entête
  	\renewcommand{\footrulewidth}{0.4pt}%trait horizontal pour les pieds de pages

\usepackage[french]{nomencl}
\makenomenclature
\usepackage{hyperref}
\usepackage{mathtools}
\begin{document}
\pagestyle{fancy}

\begin{center}\bfseries\Huge
COMPTE RENDU DE RÉUNION
\end{center}

\textbf{Du      :} jeudi 13/04/2017 à 17h00

\textbf{Objet   :} Avancement projet VITAMEAL

\textbf{Présents:} Nicolas SYMPHORIEN, Sonia OTHMANI, Jean-Félix BENITEZ

\textbf{Absent :} Personne.

\textbf{Diffusion:} Nicolas SYMPHORIEN, Sonia OTHMANI, Jean-Félix BENITEZ

\hrulefill

\section{Évaluation des besoins}
Nous avons ajouté dans le rapport les besoins évalués par Sonia; il reste à faire une petite mise en forme et à remonter sous GitHub, c'est Jean-Félix qui s'en charge ce soir.
Nous avons convenus de mettre dans une annexe toutes les énumérations métiers; c'est Nicolas qui s'en charge ce soir.
Sonia fera la relecture des besoins ainsi redéfinis pour vendredi soir. Jean-Félix fera la relecture du résultat pour dimanche soir.
Objectif \textbf{terminer l'évaluation des besoins cette semaine} pour passer au exigences la semaine prochaine.

\section{Exigences}
Nous prévoyons une téléconférence mercredi 19 avril à 20h30.

\textbf{Objectif:} Faire un premier bilan de notre travail sur les exigences.

\textbf{pré-requis:} Avoir extrait des besoins la liste des exigences.

\section{Divers}
Nous avons convenu d'utiliser les \enquote{issues} de GitHub pour suivre nos discutions sur les différents problèmes rencontrés.
Nous prévoyions d'utiliser les \enquote{Tags} de GitHub pour garder une trace de nos livraisons. C'est Jean-Félix qui s'en occupe.
Nicolas va revoir la hiérarchie des dossiers de notre dépôts gitHub. Il va aussi prospecter pour trouver un logiciel de gestion des exigences (suggestion: \url{http://reqchecker.eu/}).
Jean-Félix va planifier nos réunions de travail, ainsi que les jalons de livraisons.
Sonia va faire un bilan des solutions existantes, similaires à notre projet, et de ce qu'elles proposent.

\label{LastPage}
\end{document}
