%-*- coding: utf-8 -*-
\documentclass[11pt,a4paper,french,twoside,openright]{article}
\usepackage[utf8]{inputenc}
\usepackage[T1]{fontenc}
\usepackage{graphicx}%pour insérer images et pdf entre autres
	\graphicspath{{images/}}%pour spécifier le chemin d'accès aux images
\usepackage[left=2.5cm,right=2.5cm,top=2.5cm,bottom=2.5cm]{geometry}%réglages des marges du document selon vos préférences ou celles de votre établissemant
\usepackage[Lenny]{fncychap}%pour de jolis titres de chapitres voir la doc pour d'autres styles.
\usepackage{babel}
\usepackage[babel=true]{csquotes} % csquotes va utiliser la langue définie dans babel
\usepackage{LastPage}
\usepackage{mathtools}

\usepackage{fancyhdr}%pour les entêtes et pieds de pages
	\setlength{\headheight}{14.2pt}% hauteur de l'entête
        \chead{\textbf{VITAMEAL}}
        \lhead{}
        \rhead{}
	\cfoot{Formation Ingénieur Informatique en alternance - Première année}
	\lfoot{\textbf{CNAM}}%
  	\rfoot{\textbf{\thepage/\pageref{LastPage}}}
 	\renewcommand{\headrulewidth}{0.4pt}%trait horizontal pour l'entête
  	\renewcommand{\footrulewidth}{0.4pt}%trait horizontal pour les pieds de pages

\usepackage[french]{nomencl}
\makenomenclature
\usepackage{hyperref}
\usepackage{mathtools}
\begin{document}
\pagestyle{fancy}

\begin{center}\bfseries\Huge
COMPTE RENDU DE RÉUNION
\end{center}

\textbf{Du      :} jeudi 06/07/2017 à 17h00

\textbf{Objet   :} Avancement projet VITAMEAL

\textbf{Présents:} Nicolas SYMPHORIEN, Sonia OTHMANI, Jean-Félix BENITEZ

\textbf{Absent :} Personne.

\textbf{Diffusion:} Nicolas SYMPHORIEN, Sonia OTHMANI, Jean-Félix BENITEZ

\hrulefill

\section{Conception codage}
Nous avons convenu de restructurer le rapport selon le plan exposé par M BATATIA aujourd'hui. C'est \emph{Jean-Félix} qui s'en charge, chacun ensuite reprendra les parties qui luis incombes. La conception étant terminée, nous continuons le codage jusqu'à la veille de la prochaine présentation où nous mettrons à jour la documentation.

La conception de la génération des menus, au niveau du diagrammes des classes est à revoir notament à propos de l'utilisation des \textbf{DAO} $\rightarrow$ \emph{Jean-Félix}.

\section{Modèle du domaine}
Nous renommons la classe \emph{Part} en \emph{ComposantPlat}.

\section{Prochaine réunion}
Téléconférence dimanche 9 juillet à 20h30.

\end{document}
