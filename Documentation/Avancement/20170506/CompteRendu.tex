%-*- coding: utf-8 -*-
\documentclass[11pt,a4paper,french,twoside,openright]{article}
\usepackage[utf8]{inputenc}
\usepackage[T1]{fontenc}
\usepackage{graphicx}%pour insérer images et pdf entre autres
	\graphicspath{{images/}}%pour spécifier le chemin d'accès aux images
\usepackage[left=2.5cm,right=2.5cm,top=2.5cm,bottom=2.5cm]{geometry}%réglages des marges du document selon vos préférences ou celles de votre établissemant
\usepackage[Lenny]{fncychap}%pour de jolis titres de chapitres voir la doc pour d'autres styles.
\usepackage{babel}
\usepackage[babel=true]{csquotes} % csquotes va utiliser la langue définie dans babel

\usepackage{fancyhdr}%pour les entêtes et pieds de pages
	\setlength{\headheight}{14.2pt}% hauteur de l'entête
        \chead{\textbf{VITAMEAL}}
        \lhead{}
        \rhead{}
	\cfoot{Formation Ingénieur Informatique en alternance - Première année}
	\lfoot{\textbf{CNAM}}%
  	\rfoot{\textbf{\thepage/\pageref{LastPage}}}
 	\renewcommand{\headrulewidth}{0.4pt}%trait horizontal pour l'entête
  	\renewcommand{\footrulewidth}{0.4pt}%trait horizontal pour les pieds de pages

\usepackage[french]{nomencl}
\makenomenclature
\usepackage{hyperref}
\usepackage{mathtools}
\begin{document}
\pagestyle{fancy}

\begin{center}\bfseries\Huge
COMPTE RENDU DE TÉLÉCONFÉRENCE
\end{center}

\textbf{Du      :} mercredi 06/05/2017 à 20h30

\textbf{Objet   :} Avancement projet VITAMEAL

\textbf{Présents:} Nicolas SYMPHORIEN, Sonia OTHMANI, Jean-Félix BENITEZ

\textbf{Absent :} Personne.

\textbf{Diffusion:} Nicolas SYMPHORIEN, Sonia OTHMANI, Jean-Félix BENITEZ

\hrulefill

\section{Cas d'utilisations}

Nous avons convenu de terminer les cas d'utilisation et de les avoir intégrés
dans la présentation et le rapport pour le \textbf{lundi 08/05/2017}.
Nous avons convenu d'un formalisme commun pour la description textuelle des use
cases :

\begin{description}
\item[Nom:] Nom du use case
\item[ID:] Identifiant du use case
\item[Description:] Description succincte du use case
\item[Auteur:] L'auteur du use case
\item[Date:] La date de modification ou de création du use case
\item[Acteurs:] Acteurs en lien avec le use case
\item[Pré-Conditions:] Conditions obligatoires au bon déroulement du use case
\item[Scénario principal:] Le scénario du cas nominal
\item[Scénario alternatif:] Les scénarios alternatifs, couvre aussi les cas
d'erreur système
\item[Post-Conditions:] Indicateurs du bon déroulement du use case.
\end{description}

\noindent Lien utile :
\href{https://openclassrooms.com/courses/debutez-l-analyse-logicielle-avec-uml/la-description-textuelle-d-un-cas-d-utilisation}{Open
classroom - La description textuelle d’un cas d’utilisation}


\section{Définition du premier sprint}
\noindent Nous avons convenu que le premier sprint couvrirai en première analyse les use
cases suivant :
\begin{itemize}
   \item Renseigner un profil patient
   \item Composer les plats
\end{itemize}
Ce découpage sera affinée avec la fin de la réalisation des use cases. Il faut
aussi que chacun réfléchisse à quelques user stories pour
le \textbf{lundi 08/05/2017}


\label{LastPage}
\end{document}
