%-*- coding: utf-8 -*-
\documentclass[11pt,a4paper,french,twoside,openright]{article}
\usepackage[utf8]{inputenc}
\usepackage[T1]{fontenc}
\usepackage{graphicx}%pour insérer images et pdf entre autres
	\graphicspath{{images/}}%pour spécifier le chemin d'accès aux images
\usepackage[left=2.5cm,right=2.5cm,top=2.5cm,bottom=2.5cm]{geometry}%réglages des marges du document selon vos préférences ou celles de votre établissemant
\usepackage[Lenny]{fncychap}%pour de jolis titres de chapitres voir la doc pour d'autres styles.
\usepackage{babel}
\usepackage[babel=true]{csquotes} % csquotes va utiliser la langue définie dans babel

\usepackage{fancyhdr}%pour les entêtes et pieds de pages
	\setlength{\headheight}{14.2pt}% hauteur de l'entête
        \chead{\textbf{VITAMEAL}}
        \lhead{}
        \rhead{}
	\cfoot{Formation Ingénieur Informatique en alternance - Première année}
	\lfoot{\textbf{CNAM}}%
  	\rfoot{\textbf{\thepage/\pageref{LastPage}}}
 	\renewcommand{\headrulewidth}{0.4pt}%trait horizontal pour l'entête
  	\renewcommand{\footrulewidth}{0.4pt}%trait horizontal pour les pieds de pages

\usepackage[french]{nomencl}
\makenomenclature
\usepackage{hyperref}
\usepackage{mathtools}
\begin{document}
\pagestyle{fancy}

\begin{center}\bfseries\Huge
COMPTE RENDU DE TÉLÉCONFÉRENCE
\end{center}

\textbf{Du      :} mercredi 19/04/2017 à 20h30

\textbf{Objet   :} Avancement projet VITAMEAL

\textbf{Présents:} Nicolas SYMPHORIEN, Sonia OTHMANI, Jean-Félix BENITEZ

\textbf{Absent :} Personne.

\textbf{Diffusion:} Nicolas SYMPHORIEN, Sonia OTHMANI, Jean-Félix BENITEZ

\hrulefill

\section{Évaluation des besoins}
Dans un souci de simplicité, et de temps, nous avons décidé de retirer les personnes agées du périmètre du projet.
Sonia ce charge de supprimer les redits pour jeudi soir.
Nicolas ce charge de retirer des besoins tous les éléments qui sont dans l'annexe et de mettre une référence à l'annexe à la fin des besoins pour jeudi soir.

\section{Exigences}
Nous avons convenu de mettre les exigences dans un fichier XML avec les champs suivants:
\begin{itemize}
\item Numéro
\item Titre / Résumé
\item Corps
\item Version
\item Prioritée
\item Liens
\item Tests
\end{itemize}
Jean-Félix ce charge de mettre en place la structure de ce fichier ainsi que les schémas et scripts.

\section{Divers}
Sonia à fait des recherches sur les solutions existentes pour notre projet; elle va nous faire une synthèse des éléments qu'elle a trouvé.
Nous avons convenu que chacun travaillerait à la mise à jour de la présentation que nous ferons la semaine prochaine.

Prochaine téléconférence: lundi 24 avril à 20h30

\label{LastPage}
\end{document}
