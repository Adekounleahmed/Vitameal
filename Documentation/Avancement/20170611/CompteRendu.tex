%-*- coding: utf-8 -*-
\documentclass[11pt,a4paper,french,twoside,openright]{article}
\usepackage[utf8]{inputenc}
\usepackage[T1]{fontenc}
\usepackage{graphicx}%pour insérer images et pdf entre autres
	\graphicspath{{images/}}%pour spécifier le chemin d'accès aux images
\usepackage[left=2.5cm,right=2.5cm,top=2.5cm,bottom=2.5cm]{geometry}%réglages des marges du document selon vos préférences ou celles de votre établissemant
\usepackage[Lenny]{fncychap}%pour de jolis titres de chapitres voir la doc pour d'autres styles.
\usepackage{babel}
\usepackage[babel=true]{csquotes} % csquotes va utiliser la langue définie dans babel

\usepackage{fancyhdr}%pour les entêtes et pieds de pages
	\setlength{\headheight}{14.2pt}% hauteur de l'entête
        \chead{\textbf{VITAMEAL}}
        \lhead{}
        \rhead{}
	\cfoot{Formation Ingénieur Informatique en alternance - Première année}
	\lfoot{\textbf{CNAM}}%
  	\rfoot{\textbf{\thepage/\pageref{LastPage}}}
 	\renewcommand{\headrulewidth}{0.4pt}%trait horizontal pour l'entête
  	\renewcommand{\footrulewidth}{0.4pt}%trait horizontal pour les pieds de pages

\usepackage[french]{nomencl}
\makenomenclature
\usepackage{hyperref}
\usepackage{mathtools}
\begin{document}
\pagestyle{fancy}

\begin{center}\bfseries\Huge
COMPTE RENDU DE TÉLÉCONFÉRENCE
\end{center}

\textbf{Du      :} dimanche 11/06/2017 à 20h30

\textbf{Objet   :} Avancement projet VITAMEAL

\textbf{Présents:} Nicolas SYMPHORIEN, Sonia OTHMANI, Jean-Félix BENITEZ

\textbf{Absent :} Personne.

\textbf{Diffusion:} Nicolas SYMPHORIEN, Sonia OTHMANI, Jean-Félix BENITEZ

\hrulefill

\section{Cas d'utilisations}

Nous avons convenu:
\begin{itemize}
\item d'utiliser \href{http://www.lucidchart.com}{Lucidchart} pour faire nos diagrammes UML.
\item de remonter sous GitHub les différents diagrammes que nous avons fait pour compléter la conception des cas d'utilisations que nous avons en charge.
\item d'enrichir le rapport avec ces diagrammes et leur description.
\item que Sonia devait mettre à jour sa configuration de travail de manière à pouvoir faire fonctionner l'application dans l'état où elle est. Nicolas et Jean-Félix son disponible le soir à travers Skype ou TeamViewer en cas de besoin.
\item que Jean-Félix doit mettre à jour le document décrivant la configuration permettant de faire fonctionner l'application.
\end{itemize}

Références du livre utilisé pour mettre en place les premiers éléments de l'application:
\begin{description}
\item[Titre]: Java EE, Développez des applications web en Java
\item[Auteur]: Thierry RICHARD
\item[Éditeur]: Éditions eni
\end{description}

Prochain rendez-vous: \textbf{mercredi 14/06/2017 20h30 sous Skype.}

\label{LastPage}
\end{document}
