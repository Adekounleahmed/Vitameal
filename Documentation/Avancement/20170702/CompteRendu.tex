%-*- coding: utf-8 -*-
\documentclass[11pt,a4paper,french,twoside,openright]{article}
\usepackage[utf8]{inputenc}
\usepackage[T1]{fontenc}
\usepackage{graphicx}%pour insérer images et pdf entre autres
	\graphicspath{{images/}}%pour spécifier le chemin d'accès aux images
\usepackage[left=2.5cm,right=2.5cm,top=2.5cm,bottom=2.5cm]{geometry}%réglages des marges du document selon vos préférences ou celles de votre établissemant
\usepackage[Lenny]{fncychap}%pour de jolis titres de chapitres voir la doc pour d'autres styles.
\usepackage{babel}
\usepackage[babel=true]{csquotes} % csquotes va utiliser la langue définie dans babel
\usepackage{LastPage}

\usepackage{fancyhdr}%pour les entêtes et pieds de pages
	\setlength{\headheight}{14.2pt}% hauteur de l'entête
        \chead{\textbf{VITAMEAL}}
        \lhead{}
        \rhead{}
	\cfoot{Formation Ingénieur Informatique en alternance - Première année}
	\lfoot{\textbf{CNAM}}%
  	\rfoot{\textbf{\thepage/\pageref{LastPage}}}
 	\renewcommand{\headrulewidth}{0.4pt}%trait horizontal pour l'entête
  	\renewcommand{\footrulewidth}{0.4pt}%trait horizontal pour les pieds de pages

\usepackage[french]{nomencl}
\makenomenclature
\usepackage{hyperref}
\usepackage{mathtools}
\begin{document}
\pagestyle{fancy}

\begin{center}\bfseries\Huge
COMPTE RENDU DE TÉLÉCONFÉRENCE
\end{center}

\textbf{Du      :} dimanche 02/07/2017 à 20h30

\textbf{Objet   :} Avancement projet VITAMEAL

\textbf{Présents:} Nicolas SYMPHORIEN, Sonia OTHMANI, Jean-Félix BENITEZ

\textbf{Absent :} Personne.

\textbf{Diffusion:} Nicolas SYMPHORIEN, Sonia OTHMANI, Jean-Félix BENITEZ

\hrulefill

\section{Conception}
Avancement du travail de Sonia:
Diagrammes de cas d'utilisation, séquence système et séquence détaillé faits et à jours. Reste à détailler le diagramme de classes avant de passer au codage.

Avancement du travail de Nicolas:
Conception terminées dans les grandes lignes. Codage de \enquote{consulter un plat} fait, \enquote{Créer un plat} en cours; reste à coder \enquote{supprimer un plat} et \enquote{éditer un plat}.

Avancement du travail Jean-Félix:
Diagramme des classes de l'élaboration des menus fait. Description écrite de l'algorithme d'élaboration des menus fait. Reste à passer au codage.

\section{Rapport}
Nous avons convenu de renommer le chapitre actuel \enquote{Initialisation}. Le nouveau nom devra évoquer la description du problème. Nous le feront suivre par un chapitre consacré à la conception, qui ce trouvera avant le chapitre \enquote{Élaboration}. Ce changement interviendra dès que  nous aurons trouvé un nouveau nom pour le chapitre (pas forcément pour la prochaine présentation).

\section{Présentation}
On garde la structure actuelle, chacun va venir ajouter / mettre à jour les diagrammes dont il a la charge. Mettre en place des vues plein écran pour les diagrammes (sans le menu): Jean-Félix.

\section{Prochain rendez-vous}
\textbf{mercredi 05/07/2017 17h00 après les cours}

\end{document}
